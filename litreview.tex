
\pagebreak
\section{Text S2: Literature review methods}

To review the use of thermography in ecological studies we conducted a review of peer-reviewed literature in the Web of Science Core Collection. Search terms were: ``TS = (``therm\* camera'' OR ``therm\* imag\*'' OR ``therm\* photo'' OR ``infra\* imag\*'' OR ``infra\* photo'' OR thermograph\*) NOT TS = (remote sensing OR satellite OR LANDSAT) AND WC = (Biodiversity Conservation OR Ecology) NOT WC = (Remote Sensing)''. The search was run on 3\textsuperscript{rd} November 2018, and returned 118 studies. Reviewing all studies manually and excluding any that only used thermal images for animal detection left a total of 48 articles, published between 1977 and 2018 (see reference list). For comparison with all ecological studies published we repeated the seach for the same time period, excluding the thermography-specific search terms. This returned 497,494 studies. To analyse trends over time (Figure 1) we fit a binomial generalized linear model of the proportion of ecological studies that used thermography (binomial data: number of studies using thermography versus the number not using thermography) against year, using the package \textmyfont{lme4} (Bates et al., 2015) in \textmyfont{R} (R Core Team, 2018).

\nocite{*}

\subsection*{References}
\addcontentsline{toc}{subsection}{References}

Bates, D., Mächler, M., Bolker, B., Walker, S., 2015. Fitting Linear Mixed-Effects Models Using lme4. Journal of Statistical Software 67, 1–48. doi:10.18637/jss.v067.i01

R Core Team, 2018. R: A Language and Environment for Statistical Computing.

\renewcommand\bibsection{\subsection*{\refname}}
\addcontentsline{toc}{subsection}{Thermography literature}
\renewcommand\refname{Thermography literature}
\bibliographystyle{apalike}
\bibliography{thermog_studies}
